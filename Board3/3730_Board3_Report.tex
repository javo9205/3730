\documentclass{article}
\usepackage[letterpaper,margin=0.5in]{geometry}
\usepackage[sfdefault]{roboto}
\usepackage[T1]{fontenc}
\usepackage[dvipsnames]{xcolor}
\usepackage{tabularx}
\usepackage{parskip}

\newcommand*{\Indent}{\hspace*{1cm}}

\begin{document}
\setlength\parindent{0pt}

%=============================================================
% Header
%=============================================================
\noindent
\rule{\textwidth}{2.5pt}\\
\begin{tabularx}{\textwidth}{>{\bfseries}l >{\bfseries\centering}X >{\bfseries}r}
ECEN 3730   & Practical PCB Design \& Manufacture & Professor Tim Swettlten\\
Spring 2024 & Board 3 Report                      & James K Vogenthaler
\end{tabularx}\\
\rule{\textwidth}{1.5pt}

%=============================================================
\section*{Plan of Record}
%=============================================================
\subsection*{Objective}
The \textit{Golden Arduino} will be a custom redesign of the popular Arduino R3 utilizing best design practices to maximize power and signal integrity (PI/SI).  This entails employing a combination of design choices such as:
\vspace*{2mm}\\
\Indent (1) The placement of decoupling capacitors next to all power pins;\\
\Indent (2) A continuous ground plane to minimize the impedance of return paths;\\
\Indent (3) The placement of ground vias to minimize return paths wherever possible.
\vspace*{2mm}\\
In particular, special attention will paid toward the differences in switching noise and near-field emissions between this board's design and a commercial Arduino R3 board design using a specialized "noise shield" which has been engineered to draw a large amount of current in a short amount of time.

\Indent Finally, features will added expressly for risk mitigation or bring up testing.
A TVS IC will be introduced to protect sensitive components from the effects of electrostatic discharge (ESD).
Additionally, isolation switches, test points, and indicator LEDs will be placed throughout the design to aid in debugging and bring up testing.

%-----------------------------------------
\subsection*{Introduction}
%-----------------------------------------
Switching noise is a phenomenon seen along a or between nearby signal traces due to inductive or capactive cross-talk. 

\Indent The first type of noise that this design seeks to mitigate is known as \textit{power rail collapse}, and occurs when there is a sudden draw of current on the power rail.
The change in current over time (dI/dt) that passes through the power rail's parasitic inductance temporarily causes a voltage drop, and similarly a voltage spike can be observed when the current suddenly stops, which is then passed along to any accompanying load.
This type of switching noise can be dramatically reduced through the use of decoupling capacitors to decouple nearby loads from the inductance of the supplying trace.

\Indent The second type of noise this design seeks to mitigate is known as \textit{ground bounce}, and similarly occurs due to changing currents passing through the parasitic inductance of the signal's return path, and can is compounded with each signal that shares this return path. 
This kind of noise can be reduced through the usage of a solid ground plane, which both minimizes the impedance of and distributes the return path of any connected signals.

\Indent Finally, the third type of noise this design seeks to mitigate are the range of \textit{near field emissions} which can extend much farther than intended when there is a large separation between signals and their return paths.
All interconnects will exhibit some near field electromagnetic interference, but their effect can be reduced through the placement of ground vias wherever possible.

%-----------------------------------------
\subsection*{Component Selection}
%-----------------------------------------
- ATMEGA328P
- 5.0V and 3.3V LDO (Low Dropout regulators)
- 

\section*{Design, Assembly, \& Bring Up}

\section*{Analysis \& Conclusions}

\end{document}
